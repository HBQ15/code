\documentclass{article}
\usepackage{graphicx} % for figures
\usepackage{float}
\usepackage[export]{adjustbox}
\usepackage{fancyhdr}
\begin{document}

\title{Homework 7 - Physics 240\\
		Wilberforce Pendulum}
\author{Tin Tran}

\maketitle

\section{Introduction}
This is an excercise to calculate the equation of motions $z(t)$ and $\theta(t)$ of the Wilberforce Pendulum using the forth-order Runge Kutta method.\\
\indent The Lagrangian for this system is : \\
\begin{center} $L = \frac{1}{2}m(\frac{dz}{dt})^2 + \frac{1}{2}I(\frac{d\theta}{dt})^2 - \frac{1}{2}kz^2 - \frac{1}{2}\delta\theta^2 - \frac{1}{2}\epsilon z\theta$\\
\end{center}
From this, the equations of motion are:
\begin{center}
$mz^" + kz + \epsilon z\theta = 0$\\
$I\theta^" + \delta\theta + \frac{1}{2}\epsilon z = 0$
\end{center}
Now we apply the Fourth-order Runge Kutta method to approximate $z(t)$ and $\theta(t)$

\noindent ** I couldn't not finish the homework in time, I have most of the code done, but I didn't have enough time to fix all the bugs, because the graph doesn't show anything, and for some reason my code doesn't update the values of z and theta if the initial condition is 0 and 0.

\end{document}